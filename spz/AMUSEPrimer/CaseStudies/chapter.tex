%%%%%%%
% Simon Portegies Zwart
% Version 2.0 February 2012
%
% The Art of Computational Astrophysics
% Graduate School in Computational Astrophysics
%%%%%%%

%-------------------------------------------------------------------------
%-------------------------------------------------------------------------
\Chapter[Monolythic solvers]
        {Monolythic solvers}
        {}
%       {S. F. Portegies Zwart\\x
%         Sterrewacht Leiden
%       }

%-------------------------------------------------------------------------
\section{Case Studies}\label{Ch:CaseStudies:Introduction}

\subsection{Gravity with stellar evolution}
\cite{1987MNRAS.224..193T}
Run models VII and XV, reproduce figures 4 and 6


\subsection{Gravity with hydrodynamics}

Martin Heemskerk was a young graduate student at the Astronomical
Institute ``Anton Pannekoek'' in Amsterdam when he published an
important research paper \cite{1992A&A...260..161H} in which he
discussed the development of $M=1$ instabilities in a self-gravitating
gaseous discs. For this research he developed a grid-based
hydrodynamics solver in which the central star was not fixed but was
allowed to move around under the self gravity of the disk. The
hydrodynamics code he developed for this research solved the equations
for gravitational hydrodynamics on a cylindrical grid. Both ware
extremely novel and challenging for that time.

They stated their simulations with an axis symmetric disk around the
central star $r=0$ with polytropic equation of state with $\gamma =
1$.  Their density profile falls off near the stellar suface and
towards the edge of the disk. The outer edge is chosen to be 10 times
larger than the inner edge of the disk. This choise is motivated by
the numerical burden of the code to extend the disk closer to the
central star.  their initial discs are in pressure and gravitational
equilibrium.

After having generated theis stable disc, \cite{1992A&A...260..161H}
continue by introducing a small $m=1$ perturbation half-way the disk.
This perturbation causes the center of mass of the disk and the
central star to be slightly off the geometric center with respect to
eachother. As a consequence the disc and star become unstable.  The
instability is worst for a stellar mass to disk mass ratio of 0.1 to
1, and deminishes when the stellar mass approaches the disk mass.

We reproduce their results with a small AMUSE script, but there are
some profound differences in our approach. First of all we adopted an
SPH code for the hydrodynamics, and a direct N-body integration for
the single star under the infuence of the self gravitating disk.  The
gravitational back coupling between star and disk is also integrated
using the direct N-body code.

We set un the experiment by initializing a axis symmetric Keplerian
disc with a poer-law density profile between 10\,AU and 100\,AU.  We
opted for an equation of state with $\gamma=1$ and we varied the disc
mass between the same mass as the central star to 10 times this value.
So far our initial conditions are quite similar to those adopted by
\cite{1992A&A...260..161H}.

Instead of following \cite{1992A&A...260..161H} for the method of
introducing an $m=1$ perturbation in the disc, we opted for a much
simpler approach. Adopting a disc size of 10\,AU, we measure the
number of disc particles within this radius half way the disc. We
subsequently generate a Plummer distribution of sph particles with a
radius of 10\,AU and insert this half way the disc. To assure that the
disc is still in gravitational equilibrium we give each of the
injected bump particles a velocity consistent with the Keplerian
velocity at that location in the disc. This is slightly more
complicated than it sounds becuase due to the high mass of the disc we
have to account for the total mass withhin the bump's orbit to be able
to calculate its circular orbital velocity. Having done so, we center
disk (plus bump) and the central star.

Teh integration of the hydrostatic equations and Newtons' equations we
opted for two codes, Fi for the hydrodynamics and Hermite1 for the
self gravity of the disk and star. Inter module communication is
realized using bridge. BEcause we are interested in the mutual
influence of gaseous disc on the star and the star of the
self-gravitating disc we initialize a two-way bridge.

The tricky part in AMUSE is the time stepping between the
gravitational solver and the hydrodynamics solver.  After a little
experimenting we decided to have the hydro time step be be 1/10th of
the gravitational time step.

To qualify and quantify the consistency of the initial conditions and
the accuracy of the code coupling we run a 4000 particle SPH
simulation for 100 years. During this time frame the inner edge of the
disc closes in on the central star, but the disc remains stable and
static. We perform this simulation with Mdisc/Mstar = 1 and 10.  In
both cases the disc appears stable on the selected time scale.

For convenence we adopted a softening of $\eps = R_{\rm min}/N_{\rm
  disc}+N_{\rm bump}$.

python 1992AA...260..161H.py -t 100 -n 200 --Ndisk 4000 --Mdisk 1.0 --Mstar 0.1 --Rmin 10.0 --Rmax 100 &


\begin{table}[htbp]
\begin{center}
\begin{tabular}{crcccccccccc}\hline
$M_{\rm star}/M_{\rm disc}$ & $N_{\rm disc}$ & $N_{\rm bump}$  \\
1  & 4096 \\
1  & 4096 \\
10 & 4096 \\
10 & 4096 \\
\hline
\hline
\end{tabular}
\caption{\label{Tab:Timings} 
Simulation results of \cite{1992A&A...260..161H}.
 }
\end{center}
\end{table}


\cite{1994A&A...288..807H}

%\input /home/spz/latex/lib/bib/references
