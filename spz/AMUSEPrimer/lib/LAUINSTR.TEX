\documentclass{iopbk2e}
% Uncomment following line to include characters from AMS Fonts
%\usepackage{iopams}

\title{Writing a book for Institute of Physics Publishing using 
\LaTeXe}

\author{A J Cox\\[2pc]
Institute of Physics Publishing,\\
Techno House, Redcliffe Way, Bristol}

\makeindex

\newtheorem{example}{Example}

\begin{document}

\maketitle
\pagenumbering{roman}
\setcounter{page}{5}
\tableofcontents

\chapter*{Preface}
This article describes the use of \LaTeXe\ and the `iopbk2e.cls' 
class file for the preparation of books to be 
published under the Institute of Physics Publishing imprint. 
A separate style file (iopbk.sty)
is available for authors still using \LaTeX 2.09.
Authors who use these systems
correctly should find that the writing of their book is
easier, since the output will be in a form directly suitable for 
camera-ready copy or for direct automatic conversion into the Times 
fonts usually used for typesetting. 
In addition, the book should be produced more quickly, and 
with less chance of typographic error, since it will already be
in house style. 

After a brief introduction in chapter 1 to 
the class files and the method of 
preparing the book, we discuss in chapter~2 the 
preparation of the preliminary pages. In chapter~3 we consider the 
text and its basic structures.
In chapter~4 we discuss the use of \LaTeXe\ in typesetting mathematics 
and in 
chapter~5 we present the rules for preparing figures and tabular 
material. Chapter 6 contains information on references and reference 
lists while chapter 7 contains a description of the tools available 
for preparing indexes, 
and the final chapter discusses the procedures that will be 
followed in publishing the book. In the appendices we give a list of 
author-usable macros.

The style files have been developed at IOP Publishing mainly by Tony 
Cox, but building on earlier work by Robert Harding and Doug Quinney of
Keele University and Mark Manning of Cambridge University. The translation
from \LaTeX2.09 to \LaTeXe\ was first done by Steve Lloyd of Queen Mary \&
Westfield College, University of London. 

\bigskip
\begin{flushright}
{\bf A J Cox}\\
17 February 1997
\end{flushright}
\newpage

\pagenumbering{arabic}
\setcounter{page}{1}

\chapter{Introduction}
This document describes the preparation of a book for the Institute of 
Physics Publishing imprint
using the \verb"iopbk2e"\index{iopbk2e.cls} class file for \LaTeXe.
It covers how to prepare your source code, the house style for 
Institute of Physics Publishing (IOPP) and also the procedures 
followed during the production of the book. 
If read carefully and adhered to, it should
result in trouble-free editing and more rapid publication, as well as 
ensuring that the final product is of a high quality. 
Our booklet {\it Notes for Authors}
contains more general information related to preparing a 
typescript which is still relevant when using \LaTeXe. 

\section{The IOP book class files}
The book class files can be obtained from the Institute of Physics World Wide Web
({\tt http://www.iop.org/Books/Author/texstyle.html}), by e-mail or
on PC or Mac disk.
The files should be placed in the current working directory or the
directory that contains your other
\LaTeX2e\ class files, e.g.\ \verb"book.cls" and \verb"bk10.clo". 
The files\index{class files}  contain commands to 
set the page size, running heads, section and subsection headings, 
appendices and references. There are also a number of miscellaneous 
commands that can be used to simplify the typing process. 

The class file \verb"iopbk2e.cls"\index{iopbk2e.cls} 
is designed for use with \LaTeXe, a style for the old version of \LaTeX\
(\LaTeX2.09) is also available (\verb"iopbk.sty")\index{iopbk.sty}. 

This document has been prepared using the iopbk2e class file 
and so illustrates its use and demonstrates the output produced. 
The complete book from 
preliminary pages to index (apart from the actual figures) can be 
produced using \LaTeXe\ with the files provided. Staff at IOP 
will be glad to answer any queries you have relating to them, 
their use, or any other aspects of producing your book.

We strongly recommend that authors should use iopbk2e, 
rather than any other class files 
that are available. The style has been designed to produce 
a book in the correct size and to a particular format. If authors use a 
style file that produces text according to a different page layout, 
this creates extra work at IOP and can result in 
difficulties and delays in the production of the book.

Authors may add their own macros 
at the start of an article 
provided they do not overwrite existing definitions and 
that they send copies of their new macros with their files.
\verb"iopbk2e" can be used with other package files such
as those loading the AMS extension fonts 
\verb"msam" and \verb"msbm" (these fonts provide the 
blackboard bold alphabet 
and various extra maths symbols as well as symbols useful in figure 
captions); an extra style file \verb"iopams.sty" is provide to load these
packages and provide extra definitions for bold Greek letters and certain
other symbols.

This document is not intended to be a comprehensive guide to \LaTeXe. 
Authors will need to be familiar with \LaTeX\ and should 
refer to {\em \LaTeX: a Document Preparation System\/} (Lamport 1986) and 
{\em The \LaTeX\ companion\/} (Goossens {\it et al\/} 1994)
for more details.

\section{Preparing the book}
When writing the book, rather than creating one long file, 
all chapters\index{chapters} should be written in separate 
files. The complete book should be run from a master 
file\index{master file} which calls all chapters 
in turn or allows chapters can be processed selectively using the 
\verb"\includeonly" command.
It is recommended that until the book is complete  
the page number\index{page numbers} and chapter number\index{chapter
numbers} should be set manually 
within the master file, 
using, for example, \verb"\setcounter{chapter}{2}" and
\verb"\setcounter{page}{35}". 
When the book is complete most of these
counter resettings can be removed and the book processed as a whole
to generate the complete contents list and 
index.
Detailed instructions for the preparation of the book, including notes 
on preparing text, mathematics, reference lists, index, tables and figures, 
are provided in the following chapters.

The first line of the file should be \verb"\documentclass{iopbk2e}"
and then \verb"\title{#1}" and \verb"\author{#1}".
The iopbk2e style produces by default a book for a 
Royal Octavo page size with a text
area, excluding running heads and footlines, of 4.5 in by 6.3 in (114 mm
$\times$ 160 mm); this is
the normal size for a monograph. 
If this is inappropriate for a particular book, the text area can be
changed by adding an option in square brackets in the documentclass line
immediately before \verb"{iopbk2e}"
(e.g. adding \verb"[Crown]" produces a
Crown Quarto book with a text area of 5.8 in by 8 in (148 mm
$\times$ 203 mm) or \verb"[Demi]"
gives a Demi Octavo book with a page area of 4 in by 6.6
in (102 mm $\times$ 169 mm)). If your book is to be in one of these alternative sizes you will be 
informed of the appropriate format by the commissioning staff.

The argument of \verb"\title" is the title of the book\index{book title} 
and that of
\verb"\author" the list of authors\index{authors}  and their
address(es)\index{addresses}.
Any new definitions specific to your book can then be entered before the
\verb"\begin{document}" command is given to start the book or included in
a separate file that is read in at this point .
An example of a master file for a book with five chapters and an appendix
might be:

\begin{verbatim}
\documentclass{iopbk2e}
\title{Physics in the 1980s}
\author{T J Smith\\[2pc]
Department of Physics\\
University of Bristol}
\makeindex
%\includeonly{chap1}
\begin{document}
\maketitle
\pagenumbering{roman}
\setcounter{page}{5}
\tableofcontents
\include{preface}
\pagenumbering{arabic}
\setcounter{page}{1}
\include{chap1}
\include{chap2}
\include{chap3}
\include{chap4}
\include{chap5}
\include{appendix}
\include{refs}
\end{document}
\end{verbatim}
Using this method it is not necessary to specify the preamble at 
the start of each chapter and only one \verb".toc" file is generated
although until the book is complete the page 
number and chapter numbers would need to be specified 
before each chapter is loaded.

\chapter{Preliminary pages}
The preliminary pages\index{preliminary pages} 
consist of half-title page, series page (if the 
book is in a series), title page, 
copyright page, contents, preface, and, if required,
foreword\index{foreword},
dedication\index{dedication}, 
acknowledgments\index{acknowledgments} and list of symbols. 
We will normally prepare the 
first four 
of these pages ourselves and obtain the appropriate cataloguing 
information to appear on the copyright page, so authors need not 
worry about the exact format of these pages. 
The \verb"\maketitle" command can be
used to produce a sample title page\index{title page} 
provided that \verb"\title"\index{book title} and
\verb"\author"\index{authors} 
are defined in the preamble. The \verb"\author" command should include the
author(s) names and, separated by \verb"\\[2pc]", the author's
address\index{addresses}.
If the address is too long to fit nicely in one line it should be split
using \verb"\\".

Authors are responsible for the 
pages containing the contents list\index{contents list}, 
preface\index{preface} and any dedication or 
acknowledgment. This 
information should go in a separate file or files (perhaps called 
\verb"prelims.tex").  
The preliminary pages are numbered using lower case roman numerals
but the first four are normally unnumbered.
Thus the commands \verb"\pagenumbering{roman}"
and \verb"\setcounter{page}{5}" have to be given before
starting the contents list.
The entries for the 
contents list are generated automatically from the chapters, 
sections and subsections
if the \verb"\tableofcontents"\index{table of contents} 
command is given at the start of the file. 
The contents list will include part titles and authors, 
chapter titles, section 
and subsection headings; subsubsection headings are not 
normally included. The \verb".toc" file produced by the
\verb"\tableofcontents" command may need to be 
slightly edited, e.g.\ to
avoid bold symbols, before the final version is produced.
Page numbers will be automatically included; 
they may well 
change during production of the book, so the contents list
need not be produced until the rest of the book is complete.
When running a complete book the file must be run through \LaTeXe\ {\bf
three} times to generate an accurate table of contents.

The preface\index{preface} should appear on a separate page. 
The preface and similar sections such as dedication and acknowledgments
are treated as unnumbered chapters and specified by the command
\verb"\chapter*{title}", 
where \verb"title" is the name of the section, e.g.\ Preface,
Acknowledgments or Dedication. 
A preface should contain personal details about how the book came to be 
written, a description of the plan of the book, and acknowledgments if 
a separate section is not used. 
An {\it introduction\index{introduction}\/} that is an 
integral part of the actual subject matter of the book should not be 
included in the preliminary pages but rather as the first chapter of 
the book. The preface should end with the commands:
\begin{verbatim}
\bigskip
\begin{flushright}
{\bf Author's name}\\
Date preface written
\end{flushright}
\newpage
\end{verbatim}

Acknowledgments can be included in the preface, or can be written in a 
separate section if desired. If acknowledgments\index{acknowledgments} 
are to appear in a 
separate section, type the command \verb"\chapter*{Acknowledgments}"
and follow this 
with the text of the acknowledgments. 

After the preliminary pages have been completed the page numbering has to
be changed from small roman numerals to arabic numbers and the page number
reset so that the first page of the body of the text is page 1. This is
done with the commands
\verb"\pagenumbering{arabic}" and
\verb"\setcounter{page}{1}"\index{page numbers} 
which should appear in the master file
before the first chapter of the main body of the book.

\chapter{The text}
\section{The main body of text}
\subsection{Parts}
Some books may be subdivided into 
distinct parts\index{parts} perhaps by different 
authors. Each part is then further subdivided into chapters and 
sections as normal. A part heading always occurs on a right-hand page 
and so is always an odd-numbered page. It is always followed by a 
blank page. The part heading should be included in the file for the 
chapter following it. In the IOP style files 
there are two commands which can
be used for part headings depending on whether or not an author and his
address are given. The command \verb"\part*{#1}" is used for a part heading
without an author's name with \verb"#1" the title of the part. 
If the author's name is given but no address 
then the appropriate command is \verb"\part[#1]{#2}{#3}" where 
\verb"[#1]" is an optional argument to provide a non-standard 
table of contents
entry, \verb"#2" is the part title and \verb"#3" the authors name.
If the part heading is also to include the author's address then
the alternative command \verb"\Part[#1]{#2}{#3}{#4}"
is used. The parameters \verb"#1", \verb"#2" and \verb"#3" are as 
for \verb"\part" with \verb"#4"
the author's address.

The title 
appears in bold capitals and the author's name and address in italic.
Examples of the code for parts are
\begin{verbatim}
\Part{The structure of the atom}{P Smith}{Department of Physics,
University of Cambridge}
\end{verbatim}
and
\begin{verbatim}
\part*{Two-body systems}
\end{verbatim}



\subsection{Chapters\index{Chapters}}
Each chapter should normally be in a separate file.
If necessary the chapter number and page number can be set with
\verb"\setcounter{chapter}{<no>}" and \verb"\setcounter{page}{<no>}"
respectively. The heading at the start of the 
chapter is typeset by using the 
\verb"\chapter[#1]{#2}" command, where \verb"[#1]" is an optional
short title to be used as a running head if the main heading is too long
and \verb"#2" is the full title of the chapter. The full version of the
title is used in the table of contents and the shortened form only in the
running head.
The chapter titles should be typed with an initial capital letter only
and the rest in lower case.

Unnumbered chapters\index{chapters, unnumbered} 
such as Preface, Acknowledgment, References are 
set using the alternative form \verb"\chapter*[#1]{#2}". In this style all 
unnumbered chapters will have an entry in the table of contents 
and an alternative shortened form can also be supplied for the running head.  

\subsubsection{Multiauthor books\index{Multiauthor books}}
Where each chapter of a book is written by a different author 
then the name of the author has to appear at the start of each 
chapter. This is accomplished by using the command
\verb"\Chapter[#1]{#2}{#3}" where \verb"#1" is 
optional short title, \verb"#2" the chapter title as before,
and \verb"#3" is the author name(s) and address if necessary.
If the author's address is given it follows after the author's name
separated by a \verb"\\". Only an abbreviated address or the name of the
institution where the author works should be given; full addresses may be
presented in a list of contributors at the end of the book if necessary. 
An example of a chapter heading with an author name
and address is:
\begin{verbatim}
\Chapter[Quantum mechanics]{An introduction to the theory
of quantum mechanics}{I M Boringham\\University of Ware}
\end{verbatim}
For an unnumbered section heading with an 
author the alternative \verb"*" form
of \verb"\Chapter" is used, e.g.
\begin{verbatim}
\Chapter*{Foreword}{A Einstein}
\end{verbatim}

\subsection{Sections\index{Sections} and subsections\index{subsections}}
The \verb"\section[#1]{#2}" command is used 
at the start of each section. It will 
automatically generate the correct section number. The section 
heading\index{section 
heading}, \verb"#2",
should be typed with an initial capital then lowercase letters.
The section heading is used 
as the running head\index{running heads} on odd-numbered pages; 
if the full
title is too long to be used then a short form is added in square brackets
before the full title. 


There are \verb"\subsection{#1}" and \verb"\subsubsection{#1}" commands to
produce subsection and subsubsection headings. These work in a similar way to
\verb"\section", although there is no optional parameter for the running
head. Do not put a full stop at the end of sections, subsections or
subsubsections\index{subsubsections} headings. The headings of subsections
and subsubsections  should always have an initial capital followed by 
lowercase letters, rather than capitals throughout.

Chapter\index{cross referencing, chapters} and section\index{cross
referencing, sections}  headings can be given 
a label using the command \verb"\label{label}" and 
this can be referred to in the text using \verb"\ref{label}".
We have also defined \verb"\cref{label}" which produces
\verb"chapter~\ref{label}" and \verb"\sref{label}" which produces 
\verb"section~\ref{label}". 
There are also versions with an initial capital
(\verb"\Cref" and \verb"\Sref") which give initial capitals to chapter and
section and which should only be used at the start of a section.

For an unnumbered section, for instance a reference section if the
references are to appear at the end of each chapter, an asterisk is added
immediately after the control sequence, e.g.\ \verb"\section*{references}". 

\subsection{Some matters of style}
It will help your readers if your article is written in a clear,
consistent and concise manner. Copy preparation at Institute of Physics
Publishing will try to make sure that your work is presented to its
readers in the best possible way without sacrificing the individuality of
your writing.

The main elements of consistency and style we look for 
are presented in the booklet 
{\it Notes for Authors} (available upon request from Institute of Physics 
Publishing, 
Techno House, Redcliffe Way, Bristol, BS1 6NX, UK). Some recommended 
points to note, however, are the following.
\begin{itemize}
\item Authors are often inconsistent in the use of `ize' and `ise' endings.
We recommend using `-ize' spellings (diagonalize, 
renormalization, minimization, etc) but there are some common 
exceptions to this, for example: devise, 
promise and advise.

\item English spellings are preferred (colour, flavour, behaviour, 
tunnelling, artefact, focused, focusing, fibre, etc). We write of a 
computer program on disk; otherwise, we use `programme' and `disc'.

\item Compound words beginning `non-' or `self-' are easier to read and
understand when hyphenated (non-existent, self-consistent, etc).

\item The words table, figure, equation and reference should be written 
in full and {\bf not} contracted to Tab., fig., eq. and ref.

\item The contractions\index{contractions} 
`i.e.' and `e.g.' should appear roman, {\bf not} 
italic, and when followed by a space it should be a control space 
(\verb"\"{\tt\char'040} ).

\item Text of computer programs or input should be set verbatim in
typewriter font using the \verb"\begin{verbatim}" \dots
\verb"\end{verbatim}" commands for displayed material or
\verb+\verb"<text>"+ for in-line material (note the double quotes
surrounding the text may be
replaced by any matching pair of characters except \verb"*"). 
\end{itemize}

It will help the copy preparation and avoid unnecessary errors if you
carefully check your work for accuracy, consistency and clarity before
submission. Remember that your book will probably be read by many
people whose native language is not English and who may not therefore 
be aware of many of the subtle meanings of words or idiomatic phases
present in the English language. It therefore helps if you try and keep
sentences as short and simple as possible.

While we recommend authors to follow these guidelines, there may be some
cases where they are not appropriate (for example books aimed
specifically at the North American market); in these cases authors
should discuss their proposed style changes with the Commissioning Editor. 

\section{Appendices} 
If desired, appendices\index{appendices} can be added
to the end of the book, with material that is considered necessary to the
understanding of the subject matter, but not sufficiently important to be
an integral part of the text. 

Appendices are treated like chapters but with letters to distinguish 
between them rather than numbers, e.g.\ Appendix~A, B, C etc not Appendix
1, 2, 3. To signify the start of the appendices 
\verb"\appendix" is used, it  resets
the chapter number so that \verb"\chapter{#1}",
\verb"\section{#1}", \verb"\subsection{#1}" will now produce the
equivalent parts for an appendix, with the appendix letter instead of the
chapter number.   The typical code
for the start of an appendix  would be:
\begin{verbatim}
\appendix
\chapter{The title of appendix A}
\section{Section heading if required}
\end{verbatim}

When running appendix files separately rather than
with the rest of the book, the appendix letter can be set at the start 
of the file (but after the \verb"\appendix" command), 
using the command \verb"\setcounter{chapter}{#1}". 
The \verb"#1" parameter is  
the number corresponding to the appendix letter, e.g.\ 1 gives A, 2
gives B, etc. 

\chapter{Mathematics}
\section{Two-line constructions}
The great advantage of \TeX\ and \LaTeX\ 
over other text processing systems is their
ability to handle 
mathematics\index{mathematics} to almost any degree of complexity. 
\LaTeX\ gives the author considerable freedom in encoding 
mathematics; however, in order to produce output that is of book 
quality, authors should exercise some restraint on the constructions 
used.

For instance, displayed equations\index{equations} 
should generally be restricted to a 
height of two lines, e.g.\ constructions such as 
\[ 
P={{\displaystyle{a\over b}+{c\over d}+{b\over c}}\over 
(a^2+b^2)(c^2+d^2)} 
\] 
should not be used. Instead, they should be converted to the 
equivalent two-line forms 
\[ 
P={a/b+c/d+b/c\over (a^2+b^2)(c^2+d^2)} 
\] 
or 
\[
P=\left({a\over b}+{c\over d}+{b\over c}\right)
[(a^2+b^2)(c^2+d^2)]^{-1}. 
\] 
For simple fractions\index{fractions} 
in the text the solidus \verb"/", as in 
$\lambda/2\pi$, should be used instead of \verb"\frac" or \verb"\over", care 
being taken to use parentheses where necessary to avoid ambiguity, for 
example to distinguish between $1/(n-1)$ and $1/n-1$. Exceptions to 
this are the proper fractions $\frac12$, $\frac13$, $\frac34$, 
etc which are better left in this form. In displayed equations 
horizontal lines are preferable to solidi provided the equation is 
kept within a height of two lines. A two-line solidus should not be 
used; the construction $(\ldots)^{-1}$ should be used instead.
The command \verb"\case{#1}{#2}" will give a text style fraction in a 
displayed equation and should be used where fractions such as 
$\frac{1}{2}$ are the only two-line entities within a display.

\section{Roman and italic in mathematics}
In maths mode \LaTeX\ automatically sets variables in an italic 
font. In most cases authors should accept this italicization. However, 
there are some cases where it is better to use a Roman 
font\index{mathematics, roman in}; for 
instance, in IOPP books we use a Roman d for a differential d, a Roman e 
for an exponential e and a Roman i for the square root of $-1$. To 
accommodate this and to simplify the  typing of equations we have 
provided some extra definitions. \verb"\rmd", \verb"\rme" and \verb"\rmi" 
now gives Roman d, e and i respectively for use in equations, 
e.g.\ $\rmi x\rme^{2x}\rmd x/\rmd y$ 
is obtained by typing \verb"$\rmi x\rme^{2x}\rmd x/\rmd y$". 
\verb"\rme" gives a Roman e for use in simple 
exponentials such as $\rme^x$. 

Certain other common mathematical functions, such as cos, sin, det and 
ker, should appear in Roman type. \LaTeX\ provides macros for 
most of these functions 
(in the cases above, \verb"\cos", \verb"\sin", \verb"\det" and \verb"\ker" 
respectively), we have also provided 
additional definitions for $\Tr$, $\tr$ and 
$\Or$ (\verb"\Tr", \verb"\tr" and \verb"\Or", respectively). 

Subscripts and superscripts should be in Roman type if they are labels 
rather than variables or characters that take values. For example in the 
equation
\[
\epsilon_m=-g\mu_{\rm n}Bm
\]
$m$, the $z$ component of the nuclear spin, is italic because it can have 
different values whereas n is Roman because it 
is a label meaning nuclear ($\mu_{\rm n}$ 
is the nuclear magneton).


\section{Alignment of mathematics}
IOP books house style is to centre 
mathematical expressions, as \LaTeX\ does
by default\index{equations, displayed}\index{displayed equations}. 
When splitting equations over more than one line, the \LaTeX\ 
environments \verb"eqnarray" or \verb"eqnarray*" can be used. 
However, if an equation is so long that it occupies two or more lines, 
some thought should be given to the correct place at which to break 
the equation. The author is the best judge of how to break an 
equation, since break positions may depend on subtleties of the 
mathematics\index{mathematics} involved, but there are 
two rules that can be stated.
\begin{enumerate}
\item Displayed equations should always be broken {\it before} binary 
operators and relations, i.e.\ if we break at a `+' sign we begin the 
second line with a `+' rather than end the first line with a `+'. 
Furthermore, when an equation is broken before a binary operation the 
second line should start at least two quads to the right of where the 
innermost subformula containing the binary operation begins on the 
first line. (See example~1 below.) A clear discussion of these 
problems in found in (Knuth, pp~195--7).

\item If an equation {\it must} be split in the middle of a term (for 
example, between the two bracketed expressions in the term 
\verb"[(a+b)(c+d)]") then a multiplication sign should be inserted at the 
beginning of the second line. (See example~2 below.)
\end{enumerate}

\begin{example} (From Knuth (1986), p~195.)
The equation
\[
\sigma(2^{34}-1,2^{35},1)
=
-3+(2^{34}-1)/2^{35}+2^{35}\!/(2^{34}-1)
+7/2^{35}(2^{34}-1)-\sigma(2^{35},2^{34}-1,1).
\]
is too long and must be split. In an equation like this, good break 
points might be immediately before the equals sign, immediately 
before any of the `+' signs or immediately before the penultimate `$-$' 
sign. Bad break points would be immediately before all other `$-$' signs 
(because they are within a subformula rather than linking two parts of 
a formula) and the `\verb"/"' signs.

Following Knuth, we choose to break just before `+7'. One way of doing 
this is to type
\begin{verbatim}
\begin{eqnarray*}
\sigma(2^{34}-1,2^{35},1)
&=&-3+(2^{34}-1)/2^{35}+2^{35}\!/(2^{34}-1)\\
& & +7/2^{35}(2^{34}-1)-\sigma(2^{35},2^{34}-1,1).
\end{eqnarray*}
\end{verbatim}
which yields
\begin{eqnarray*}
\sigma(2^{34}-1,2^{35},1)
&=&-3+(2^{34}-1)/2^{35}+2^{35}\!/(2^{34}-1)\\
& &+7/2^{35}(2^{34}-1)-\sigma(2^{35},2^{34}-1,1).
\end{eqnarray*}
\end{example}


\begin{example} The equation
\[
T_{nk\rightarrow n'k'} \propto (a+b)^2
\bigg\vert \sum_{\alpha'\nu'\alpha\nu} C_{\alpha'\nu'}^{n'\star}(k)
C_{\alpha\nu}^{n}(k)\sum_l \exp{\rmi k(R_l+\tau_{\nu\nu'})} 
E_{\alpha\alpha'}^{l\nu\nu'} \epsilon(R_l+\tau_{\nu\nu'}) \bigg\vert^2
\]
should be broken thus:
\begin{eqnarray*}
T_{nk\rightarrow n'k'}&\propto& 
(a+b)^2\bigg\vert \sum_{\alpha'\nu'\alpha\nu} 
C_{\alpha'\nu'}^{n'\star}(k)C_{\alpha\nu}^{n}(k)\\
&&\times \sum_l \exp[\rmi k(R_l+\tau_{\nu\nu'})] 
E_{\alpha\alpha'}^{l\nu\nu'} \epsilon(R_l+\tau_{\nu\nu'}) 
\bigg\vert^2
\end{eqnarray*}
which is obtained by the code
\begin{verbatim}
\begin{eqnarray*} 
T_{nk\rightarrow n'k'}&\propto& 
(a+b)^2\Big\vert \sum_{\alpha'\nu'\alpha\nu} 
C_{\alpha'\nu'}^{n'\star}(k)C_{\alpha\nu}^{n}(k)\\
&&\times \sum_l \exp[\rmi k(R_l+\tau_{\nu\nu'})] 
E_{\alpha\alpha'}^{l\nu\nu'} \epsilon(R_l+\tau_{\nu\nu'}) 
\Big\vert^2
\end{eqnarray*}
\end{verbatim}
\end{example}


For series of short equations occurring consecutively several options 
are available. 
\begin{itemize}
\item If all equations fit on one line and the equations 
are not referred to separately in the text, 
then they should be set on a single 
line with a \verb"\qquad" separating each separate equation, e.g.
\begin{equation}
h_{00}=h^*_{00}\qquad h_{0i}=h^*_{0i}+a_{0,i}+a_{i,0}
\qquad h_{ij}=h^*_{ij}+a_{i,j}+a_{j,i}. \label{eq:hxy}
\end{equation}

\item If the equations will not fit on one line but do not need 
individual numbers then they should be set on the minimum number of 
lines with \verb"\qquad" separating entries (or if each individual 
equation is of similar length aligning the starts of the equations), 
e.g.
\begin{eqnarray*}
&g_{\mu\nu}=\eta_{\mu\nu}+h_{\mu\nu}\qquad
\vert h_{\mu\nu}\vert \ll 1&\\
&\eta_{00}=1\qquad \eta_{i0}=0\qquad \eta_{0j}=0 
\qquad \eta_{ij}= -\delta_{ij}.&
\end{eqnarray*}


\item If individual equations are referred to separately (i.e.\ they 
have separate equation numbers) then, if they have similar length 
left-hand sides, they should be aligned (either left or at the equals 
sign) using the \verb"eqnarray" environment.
\end{itemize}

\subsection{Special characters for mathematics}
Bold italic characters are used in our books to signify vectors (rather
than using an upright bold or an over arrow). To obtain this effect
use \verb"\bi{#1}" within maths mode, e.g. $\bi{ABCdef}$. If upright 
bold characters are required in maths use \verb"\mathbf{#1}" within maths
mode, e.g. $\mathbf{XYZabc}$.

The American Mathematical Society provides a series of extra symbol fonts
to use with \LaTeX\ and packages containing the character definitions to
use these fonts. Authors wishing to use Fraktur \ifiopams$\mathfrak{ABC}$\fi
or Blackboard Bold \ifiopams$\mathbb{XYZ}$\fi can include the appropriate
AMS package (e.g. amsgen, amsfonts, amsbsy, amssymb) with a 
\verb"\usepackage" command or add the command \verb"\usepackage{iopams}"
which loads the four AMS packages mentioned above and also provides
definitions for extra bold characters (all Greek letters and some
additional other symbols). 

The package iopams uses the definition \verb"\boldsymbol" in amsbsy
which allows individual non-alphabetical symbols and Greek letters to be made
bold within equations.
The bold Greek lowercase letters \ifiopams$\balpha \ldots\bomega$,\fi 
are obtained with the commands 
\verb"\balpha" \dots\ \verb"\bomega" (but note that
bold eta\ifiopams, $\bfeta$,\fi\ is \verb"\bfeta" rather than \verb"\beta")
and the capitals\ifiopams, $\bGamma\ldots\bOmega$,\fi\ with commands 
\verb"\bGamma" \dots\
\verb"\bOmega". Bold versions of the following symbols are
predefined in iopams, other characters are made bold using 
\verb"\boldsymbol{\symbolname}": 
bold partial\ifiopams, $\bpartial$,\fi\ \verb"\bpartial",
bold `ell'\ifiopams, $\bell$,\fi\  \verb"\bell", 
bold imath\ifiopams, $\bimath$,\fi\  \verb"\bimath", 
bold jmath\ifiopams, $\bjmath$,\fi\  \verb"\bjmath", 
bold infinity\ifiopams, $\binfty$,\fi\ \verb"\binfty", 
bold nabla\ifiopams, $\bnabla$,\fi\ \verb"\bnabla", 
bold centred dot\ifiopams, $\bdot$,\fi\  \verb"\bdot".

Table~\ref{math-tab2} lists some other macros for use in 
mathematics with a brief description of their purpose.
Both \verb"\ms" (medium space) and \verb"\bs" (big space) can be used to
provide extra spacing between lines of a displayed equation or table.
This space may be necessary when several separate equations are within the
same equation environment. 

\begin{table}
\caption{Other macros defined in IOPP macros for use in maths.
\label{math-tab2}}
\begin{tabular*}{\textwidth}{@{}l*{15}{@{\extracolsep{0pt plus
12pt}}l}}
\br
Macro&Result&Description\\
\mr
Spaces\\
\verb"\fl"&&Start line of equation full left\\
\verb"\ms"&&Spread out lines in displayed equations slightly ($\sim$3pt)\\
\verb"\bs"&&Bigger space ($\sim$6pt) to separate lines in displays\\
\verb"\ns"&&Small negative space between lines in displays\\
\bs
Miscellaneous\\
\verb"\case{#1}{#2}"&$\case{\#1}{\#2}$&Text style fraction in display\\
\verb"\Tr"&$\Tr$&Roman Tr (Trace)\\
\verb"\tr"&$\tr$&Roman tr (trace)\\
\verb"\Or"&$\Or$&Roman O (of order of)\\
\verb"\tdot{#1}"&$\tdot{x}$&Triple dot over character\\
\verb"\lshad"&$\lshad$&Text size left shadow bracket\\
\verb"\rshad"&$\rshad$&Text size right shadow bracket\\
\br
\end{tabular*}
\end{table}


\section{Miscellaneous points}
Exponential expressions, especially those containing subscripts or 
superscripts, are clearer if the notation $\exp(\ldots)$ is used, 
except for 
simple examples\index{exponentials}. 
For instance $\exp[\rmi(kx-\omega t)]$ and $\exp(z^2)$ are 
preferred to $\e^{\rmi(kx-\omega t)}$ and $\e^{z^2}$, but 
$\e^2$ 
is acceptable. 

It is important to distinguish between ln $= \log_\e$ and lg 
$=\log_{10}$. We recommend braces, brackets and 
parentheses\index{parentheses} should be used in the 
order: $\{[(\;)]\}$, but, whatever order they are 
used in, they should be consistent throughout the book. 
The same ordering of brackets should be 
used within each different size occurring. 
However, this ordering is to be ignored if the
brackets have a 
special meaning (e.g.\ if they denote an average or a function). Decimal 
fractions should 
always be preceded by a zero\index{decimals}: for example 0.123 {\bf not} .123. For long 
numbers commas are not inserted but instead a thick space (\verb"\;")
is added after 
every third character away from the position of the decimal point unless 
this leaves a single separated character: e.g.\ $60\;000$, 
0.123\;456\;78 
but 4321 and 0.7325 (\verb"\;" has been redefined to allow it to be 
used in both text and maths mode).

No punctuation\index{punctuation} should be used in 
displayed mathematics apart from full stops if 
the equation comes at the end of a sentence. If more than one equation 
appears on the same line, they should be 
separated by a \verb"\qquad" of space 
rather than by a comma or semicolon.

The symbols `$>$' and `$<$' should be used in the sense of `greater 
than' and `less than'. If an angle bracket is required (for instance 
in a bra or ket vector) do not use these symbols; instead, use 
\verb"\rangle" and \verb"\langle". 

Theorems, and other theorem-like structures such as lemmas, corollaries,
definitions, exercises and examples, should be set using the \verb"\newtheorem"
declaration to define environments for the particular theorem-like
structures in your document. There are two arguments to \verb"\newtheorem",
the first is the name of the environment and the second the text to be
printed when the environment is called. There is also an optional argument 
to specify numbering by chapter or section. For more details see Lamport
(1986) pp~58--9.
To indicate the end of a theorem 
a small square, \opensquare, is often used; to obtain one at the end of the line 
type \verb"\hfill$\square$" or \verb"\hfill\opensquare".
 


\section{Equation numbering}
Displayed equations that are referred to in the text should be numbered with 
the number on the right-hand side\index{equation numbering}. 
Sequential numbering by chapter should be 
used: (1.1), (1.2), etc. 
When referring to an equation in the text, either put 
the equation number in parentheses without the word `equation', 
e.g.\ `as in (1.2)', or spell out the 
word equation in full, e.g.\ `if equation (1.2) is factorized'; do not 
use abbreviations such as eqn or eq. `Equation' should always be used 
if an equation number starts a sentence.
\LaTeX\ generates equation numbers automatically by using the
\verb"equation" or \verb"eqnarray" environments. 
Cross referencing\index{cross referencing, equations} can be used
for equations numbers by adding a label to the equation, e.g.\
\verb"\label{eq:HF}" and referring to it in the text by
\verb"(\ref{eq:HF})", or \verb"\eref{eq:HF}" (which automatically surrounds
the number with parentheses), see \eref{eq:hxy}. Where an equation 
number starts a sentence \verb"\Eref{#1}" will produce 
\verb"Equation~(No)" where No is the number corresponding to label 
\verb"#1".




\chapter{Figures and tables}
\section{Figures}
Figures may be included in an article as encapsulated PostScript files or
using the \LaTeX\ picture environment. Alternatively authors may send in 
high quality printed versions of their figures (fair copies) and 
attach copies of the fair copies to each 
chapter. The fair copies should be in black 
Indian ink or printing on tracing paper, plastic or white card or 
paper, or glossy photographs.

Figures are numbered serially within each chapter and
referred to in the text by number (figure 2.1, etc {\bf not} Fig.\
2.1). Each figure should have a brief caption describing it and, 
if necessary, interpreting the various lines and symbols on the 
figure. As much lettering as possible should be removed from the 
figure and included in the caption\index{figure captions}. 
\LaTeX\ provides several macros 
for use in describing symbols\index{figure captions, symbols} 
and lines in figures and we have
provided some additional definitions 
(symbols for a filled triangle, a filled inverted triangle, a filled star 
and a filled diamond are only available when using the optional
file \verb"iopams.sty"):
\begin{tabbing}
\verb"\opentriangledown"\qquad\=\kill
\verb"\opencircle" \>produces\qquad\=\opencircle\\
\verb"\opensquare" \>produces \>\opensquare\\
\verb"\opentriangle" \>produces \>\opentriangle\\
\verb"\opentriangledown"\>produces\>\opentriangledown\\
\verb"\opendiamond" \>produces \>\opendiamond\\
\verb"\fullcircle" \>produces \>\fullcircle\\
\verb"\fullsquare" \>produces \>\fullsquare\\
\ifiopams
\verb"\fulltriangle" \>produces \>\fulltriangle\\
\verb"\fulltriangledown" \>produces \>\fulltriangledown\\
\verb"\fulldiamond" \>produces \>\fulldiamond\\
\verb"\fullstar" \>produces \>\fullstar\\
\fi
\verb"\dotted" \>produces \>\dotted\\
\verb"\dashed" \>produces \>\dashed\\
\verb"\broken" \>produces \>\broken\\
\verb"\longbroken" \>produces \>\longbroken\\
\verb"\chain" \>produces \>\chain\\
\verb"\dashddot" \>produces \>\dashddot\\
\verb"\full" \>produces \>\full\\
\end{tabbing}

If a figure has parts\index{parts}, these 
should be labelled ($a$), ($b$), ($c$) etc. In the text use \verb"($a$)" or 
\verb"\pt(a)" to obtain \pt(a), etc.

Unless the author can provide suitable encapsulated PostScript files
or graphics files in some other suitable format the 
fair copies of the 
figures\index{Figures} will normally be reduced 
photographically and pasted into
spaces left for them.
The amount of space\index{figures, space for} the reduced version of a figure will occupy will 
of course depend on its complexity, but will usually be in the range 
10--15~picas (pc) ($\sim1.5$--2.5~in or 4--7~cm). 
The figure reductions chosen should be 
such that the lettering on the figures is of a similar size, or slightly 
smaller than, the text. If authors are contracted to produce camera-ready 
copy they should size their figures appropriately otherwise a standard 
space (e.g.\ 2in) can be left and this will be corrected later by the 
IOP editors. 

To set the figure captions\index{figure captions} 
the  \verb"\figure" environment is used.
Between the \verb"\begin{figure}" and \verb"\end{figure}"
commands the space for the figure is specified using 
\verb"\vspace{#1}" and the figure caption with \verb"\caption{#1}".
The space required is the final size of the figure itself 
(i.e.\ when it is 
reduced); 
allowance is made in the definition for the extra space separating the 
figure from any preceding text and from the figure caption.
The argument of \verb"\caption" is just the words of the caption, 
the label (e.g.\ {\bf Figure 
3.1.}) is automatically set in the correct style. 
The default positions for a figure are [htbp] so, if possible, the 
figure will be inserted at the position it is called at, if this is not
possible it will be placed at the 
top of a page, the bottom of a page or lastly on a page by itself. 
An optional 
argument can be added it square brackets 
after \verb"\begin{figure}" to allow 
the position of a figure on the page to be 
varied\index{figures, positioning on page}; 
the available options are \verb"[h]" for putting the figure 
in the text where the environment appears. 
\verb"[b]" inserts the figure at the bottom of the current page, 
\verb"[p]" makes the figure 
occupy a full page, \verb"[t]" inserts the figure at the top of the
current or a subsequent page, or any combination of them. 
It is normally unnecessary to change the default,  \verb"[htbp]", but 
it may be useful to change from the default to improve the 
page layout when everything is completed.

As an example, the command
\begin{verbatim}
\begin{figure}
\vspace{5pc}
\caption{This is a short figure caption.} 
\end{figure}
\end{verbatim}
gives a caption with space for a figure occupying five picas.
\begin{figure}
\vspace{5pc}
\caption{This is a short figure caption.}
\end{figure}


The figure-making commands should be used when the figure is first 
mentioned in the text. Within text, figures should be referred to as 
figure~2.1 etc., not fig.~2.1. Cross referencing\index{cross referencing,
figures} to figures can be made by
giving the figure environment a label, e.g.\ \verb"\label{fig:test}" and
referring to it in the text using \verb"\ref{fig:test}". To save typing
\verb"\fref{#1}" has also 
been defined to give \verb"figure~\ref{#1}" and \verb"\Fref{#1}" to give
\verb"Figure~\ref{#1}".

If a figure will only fit on a page 
landscape\index{landscape}\index{figures, landscape}, 
i.e.\ when rotated 90$^\circ$, 
the command \verb"\figblank" (which increments the page 
number and leaves a blank page) should be included at the appropriate 
place in the file. The caption should then be 
set from a separate file to the rest of the book
with the page dimensions adjusted so that the normal 
page depth is the page width and the normal page width the page depth.  

\subsection{Inclusion of graphics files}
If graphics files are available as encapsulated PostScript (EPS) files 
(or are created within the \LaTeX\ picture environment) they may
be included within the body of the text at an appropriate point using a
standard graphics inclusion package. Authors should ensure EPS files meet
the following criteria
\begin{itemize}
\item the Bounding Box should indicate the area of the figure 
with a minimum of white space around it and not the dimensions of the
page.
\item Any fonts used should be from the standard PostScript set (Times,
Helvetica, Courier and Symbol).
\item Scanned images should be of 600 dpi resolution for line art (black
and white) and 150 dpi resolution for grayscale or colour.
\item Captions and labels (e.g.\ Figure 1) should not be included in the
EPS file although part letters (e.g.\ ($a$)) are acceptable provided they
are placed close or within the boundary of the figure. 
\end{itemize}

The precise coding required will depend on the graphics package being used
and the printer driver. We use a printer driver compatible with DVIPS but
authors should avoid using special effects generated by including verbatim
PostScript code within the \LaTeX\ file with specials other than the
standard figure inclusion ones.

Using the epsf package figures can be included using code such as:
\begin{verbatim}
\begin{figure}
\begin{center}
\epsfbox{file.eps}
\end{center}
\caption{Figure caption}
\end{figure}
\end{verbatim}



\section{Tables\label{tabsec}}
Tables\index{Tables} should be used only to improve conciseness or where the 
information cannot be given satisfactorily in other ways, such as 
histograms or graphs. Tables are often 
best arranged with more columns than rows, as in the examples given.
Tables are numbered serially within each chapter 
and referred to in the text 
by number (table 3.1, etc, {\bf not} tab. 3.1). Each table should have an 
explanatory caption which should be as concise as possible. If a table 
is divided into parts these should be labelled \pt(a), \pt(b), 
\pt(c), etc but there should be only one caption for the whole 
table, not separate ones for each part.

\LaTeX\ uses the table environment 
to set tables, the table caption\index{table caption} precedes the table 
and is set 
with the command \verb"\caption{#1}"
which has the text of the table caption as its argument. 
As with figures the default position of tables is \verb"[htbp]"
and this can be changed if necessary by adding the specific placement
instruction after \verb"\begin{table}".
The 
numbering of the captions is be done automatically and cross
references\index{cross references, tables} to 
tables in the text can be done provided a \verb"\label" is given within the
table environment. Tables with a label \verb"tab" can be referred to in the
text using \verb"table~\ref{tab}", \verb"\tref{tab}" or \verb"\Tref{tab}". \verb"\tref" sets
`table' with the number tied to it and \verb"\Tref" `Table' and the number.
\verb"\Tref" should only be used at the start of sentences.

Tables themselves are normally set using the \LaTeX\ tabular environment,
see Lamport (1986) for full details.
IOP house style for tables is simple: there is 
a bold rule at the top and bottom of the table, with a medium rule to 
separate the headings from the entries. There should be no vertical 
rules\index{rules} in the table. 
We hope that authors will follow the instructions below for setting 
tables to the IOP style but, in books, authors may use their own format
if they prefer provided the result is pleasing to the eye and
it is followed consistently throughout.

\subsection{IOP table style}
The IOP table style uses the tabular environment within the center
environment
with columns normally aligned left.
The bold rules at the top and bottom of the table are 
set with the \verb"\br" command 
(which does not need a \verb"\\" following it).
The medium rule separating the heading from the entries 
is set with \verb"\mr"
or \verb"\hline". To avoid the rules sticking out at either end of the
table add \verb"@{}" before the first and after the last descriptors,
e.g.\ \verb"{@{}llll@{}}".
 
\Tref{ex1} demonstrates the IOP table style and the code needed to
generate it is given below. Note that columns are aligned on the decimal
point.

\begin{table}
\caption{A simple example produced using the alignment commands 
described in \protect\sref{tabsec}.  The default 
alignment for all columns is left alignment.}
\label{ex1}
\begin{center}
\begin{tabular}{@{}lllllll@{}}
\br
$E_0$ (eV)&$\theta_{\text e}$ (deg)&$\lambda$&\phm$\chi$ (deg)
&\phm$\eta_1$&$\eta_2$&$\eta_3$\\ 
\mr
139.7 &60&0.74&$-21.9$&$-0.208$&0.472&146\\ 
\075.3 &75&0.75&$-15.2$&\phm---&---&---\\ 
\045.0 &90&0.60&\0$-8.12$&$-0.29$&0.41&\075\\ 
\023.57&60&0.53&\0$-4.23$&$-0.22$&0.17&\015\\
\br
\end{tabular}
\end{center}
\end{table}

Three commands have been defined to help align columns\index{tables,
column alignment} on the 
decimal point. \verb"\0" has been defined to be a phantom digit. If placed at
the start of an entry in a left-aligned column
it will move the entry to the right by the width of a digit (all
numbers 0--9 have the same width). It can also be used at the end of
entries in right-aligned columns to move entries one digit to the left.
\verb"\0" has been used in columns one and four. \verb"\phm" has been defined as a 
phantom minus sign and is used to aid alignment in columns where most
entries have minus signs but some do not. In left-aligned columns it is
inserted at the start of the entries without minus sign and produces a
space the width of a minus sign. \verb"\phm" has been used in the headings of
columns four and five and before the dash in column five.
For columns with only a few minus signs
an alternative \verb"\m" is used instead of the minus sign to overlap the minus
sign to the left of the column.

The full code for \tref{ex1} is (note that 
\verb"\protect" has to be used before
the cross-reference in the caption):
\begin{verbatim}
\begin{table}
\caption{A simple example produced using the alignment commands 
described in \protect\sref{tabsec}.  The default 
alignment for all columns is left alignment.}
\label{ex1}
\begin{center}
\begin{tabular}{@{}lllllll@{}}
\br
$E_0$ (eV)&$\theta_{\text e}$ (deg)&$\lambda$&\phm$\chi$ (deg)
&\phm$\eta_1$&$\eta_2$&$\eta_3$\\ 
\mr
139.7 &60&0.74&$-21.9$&$-0.208$&0.472&146\\ 
\075.3 &75&0.75&$-15.2$&\phm---&---&---\\ 
\045.0 &90&0.60&\0$-8.12$&$-0.29$&0.41&\075\\ 
\023.57&60&0.53&\0$-4.23$&$-0.22$&0.17&\015\\
\br
\end{tabular}
\end{center}
\end{table}
\end{verbatim}

Headings\index{column headings}\index{table headings}
which span more than one column should be set using the command
\verb"\multicolumn{#1}{c}{#3}" where \verb"#1" is the number of columns to be
spanned and \verb"#3" is the heading. A simplified alternative version is
\verb"\centre{#1}{#2}" where \verb"#1" is the number of columns to be
spanned and \verb"#2" the heading.
There should be a rule spanning the same  
columns below the heading, it is produced with \verb"\cline{n-m}" 
where \verb"n" is the number of the first spanned column and \verb"m"
that of the last spanned column. \verb"\cline" should not be part of a row 
but follow immediately after a \verb"\\". 
An alternative way of producing a spanned
rule is to include \verb"\crule{#1}" within the row following the centred
heading, with \verb"#1" the number of columns the rule is to span. To reduce
the
space between the centred heading and rule \verb"\ns" can be added after the
\verb"\\" of the row containing the centred heading, see the code for
\tref{tnote}. 
 
If a table contains notes\index{tables, notes to} 
they should appear beneath the table set to the width of the page
and 
not at the foot of the page. 
The tabular environment for a table with footnotes must be 
placed within a \verb"\notedtable{#1}" command to save the table width for 
use in the notes.
The appropriate symbols should be included 
in the body of the table and the table notes are placed
after the closing brace
of \verb"\notedtable"
but before the \verb"\end{center}" and \verb"\end{table}" commands.
The normal symbols to use for notes to tables are either the 
footnote symbols: \dag, \ddag, \S\ or $\|$ or, if there are a lot of 
notes, superscripted lower case roman letters, i.e.\ $^{\text a}$.
The table notes are set with the command \verb"\tabnote{#1}" where 
\verb"#1" is 
the note, including its symbol. There should be a \verb"\tabnote" 
command for 
each table note. \Tref{tnote} is an example of a table containing notes.

 
\begin{table}
\caption{A table with headings spanning two columns and containing
footnotes. 
To improve the visual effect a negative skip has been 
put in between the first and second lines of the headings.}
\label{tnote}
\begin{center}
\notedtable{\begin{tabular}{@{}lllll@{}}
\br
&&&\centre{2}{Separation energies}\\
\ns
&Thickness&&\crule{2}\\
Nucleus&(mg cm$^{-2}$)&Composition&$\gamma$, n 
(MeV)&$\gamma$, 2n (MeV)\\
\mr
$^{181}$Ta&$19.3\0\pm 0.1^{\text a}$&Natural&7.6&14.2\\
$^{208}$Pb&$\03.8\0\pm 0.8^{\text b}$&99\%\ enriched&7.4&14.1\\
$^{209}$Bi&$\02.86\pm 0.01^{\text b}$&Natural&7.5&14.4\\
\br
\end{tabular}}
\tabnote{$^{\text a}$ Self-supporting.}
\tabnote{$^{\text b}$ Deposited over Al backing.}
\end{center}
\end{table}

This table is obtained by using the following code.


\begin{verbatim}
\begin{table}
\caption{A table with headings ... .}\label{tnote}
\begin{center}
\notedtable{\begin{tabular}{@{}lllll@{}}
\br
&&&\centre{2}{Separation energies}\\
\ns
&Thickness&&\crule{2}\\
Nucleus&(mg cm$^{-2}$)&Composition&$\gamma$, n 
(MeV)&$\gamma$, 2n (MeV)\\
\mr
$^{181}$Ta&$19.3\0\pm 0.1^{\text a}$&Natural&7.6&14.2\\
$^{208}$Pb&$\03.8\0\pm 0.8^{\text b}$&99\%\ enriched&7.4&14.1\\
$^{209}$Bi&$\02.86\pm 0.01^{\text b}$&Natural&7.5&14.4\\
\br
\end{tabular}}
\tabnote{$^{\text a}$ Self-supporting.}
\tabnote{$^{\text b}$ Deposited over Al backing.}
\end{center}
\end{table}
\end{verbatim}





If a table is approximately of page width\index{tables, page width} 
it can be made to fill the 
full page width by using the \verb"tabular*" environment.
Where a table is still too wide to fit on a page it may be possible to 
reorganize it to fit in (for instance change rows into columns and 
vice versa). If a large table has to be 
set landscape\index{tables, landscape} then it should 
be set separately and the command \verb"\tabblank" inserted at the 
appropriate place to increment the table number and to leave a blank 
page. For very narrow tables it may sometimes be necessary to space 
out the table so that any notes can be set satisfactorily (they are set 
to the width of the table). To do this simply replace ampersands by 
double ampersands (or triple ones) to increase the spacing between 
columns.
 
It can often be difficult to decide on the correct 
\index{tables, column alignment}alignment of 
columns and unfortunately there are no hard and fast rules. Useful 
guidelines, however, are:
\begin{itemize}
\item align decimal points\index{decimal points} where possible (Knuth 1986, 
pp~241, 242);

\item columns of numbers without decimal points should align right as 
if they had decimal points;

\item dashes in tables (use an em rule, \verb"---") 
should align with the first complete column on the 
left-hand side or the entries immediately above them;

\item column headings\index{column headings} should normally be aligned left;

\item minus signs (and other mathematical signs such as +, $<$, etc) 
should be ignored for alignment purposes (i.e.\ in a 
column of numbers, some with minus signs, the minuses should stick out 
to the left); 

\item if there is a small column heading over a wide column with an 
uneven left-hand edge, the heading may look better if aligned with the 
left-hand side of the first entry in that column;

\item column headings spanning two or more minor headings should be 
centred.
\end{itemize}

It may be difficult to satisfy all these points simultaneously but the 
most important underlying point is to create a table that is easy to 
read and understand and that has a pleasing appearance to the eye.

Thus, the procedure to typeset a table is: 
\begin{itemize}
\item start the table environment
with \verb"\begin{table}",
 type the caption within 
braces after the \verb"\caption" command, and define the label
if required;
\item start the center and tabular environments
set each column as left aligned with \verb"\begin{center}"
\verb"\begin{tabular}{@{}lllll@{}}", with one \verb"l" for each column,
the \verb"@{}" stop the rules overhanging the left and right
edges of the table;
\item set the top bold rule with \verb"\br", do not 
follow it with \verb"\\";
\item type 
the various column headings separated by \verb"&"
(each line should end with a \verb"\\"); 
\item 
follow the headings with \verb"\mr" without a \verb"\\" following;
\item set the table entries as 
usual, with different columns being separated by \verb"&" and \verb"\\" at the
end of each line;
\item insert the final bold rule using \verb"\br";
\item complete the table with \verb"\end{tabular}", \verb"\end{center}" and
\verb"\end{table}". 
\end{itemize}


Units should not normally be given within the body of a table but 
given in brackets following the column heading; however, they can be 
included in the caption for long column headings or complicated units. 

Where possible, tables should not be broken over pages. If each table 
is reasonably large then each one can start a new page; however, if 
there are several consecutive short tables they may appear on the same 
page.

If a table is not to be typeset but reproduced from artwork obtained from 
another source (e.g.\ a previous edition or another book)\index{tables,
from CRC}, 
the normal table environment should be used together with \verb"\caption" 
to set the caption.
\verb"\vspace*{#1}", where \verb"#1" is the 
space required to fit the artwork in, is then used to provide the
necessary space before the table environment is closed.



\chapter{References}
\section{General remarks}
Two different styles of referencing are in common use: the 
Harvard 
alphabetical system\index{references, Harvard 
alphabetical system} and the Vancouver numerical 
system\index{references, Vancouver numerical 
system}. Either may be 
used. References\index{References} can be collected 
together at the end of the book or can be placed at the end of each 
chapter as a numbered or unnumbered section. 
If the 
references are to be placed at the end of the book they should appear 
in a separate file with  the command
\verb"\References" or \verb"\Bibliography" at the beginning depending
on whether the heading is to be `References' or
`Bibliography', or indeed \verb"\chapter*{Title}" if some other title is
required for the reference chapter. 
If the references are to appear at the end of each chapter 
then they should be 
included at the end of the appropriate chapter file following a
command \verb"\section*{heading}" for an unnumbered section, or 
\verb"\section{heading}" for a numbered one. In each case introductory
text may
precede the same of the reference list if desired.

If the Harvard alphabetical 
system\index{references, Harvard alphabetical system} is to be used the command 
\begin{verbatim}
\begin{thereferences}
\end{verbatim}
must 
precede the start of the references (note 
\verb"\begin{thebibliography}" is reserved for use with numeric reference
lists and
\verb"\begin{thereferences}" should only be used for alphabetic
reference lists). 

The equivalent command for the Vancouver numerical reference
system\index{references, Vancouver numerical 
system} is
\verb"\begin{thebibliography}{#1}", where \verb"#1" is a number 
with the same number of digits as the highest referencing number used. 
Thus for a reference list containing 156 entries, \verb"{156}" (or 
with any other three-digit number) should be used.

The reference environments remove extra 
spaces after full stops and adjusts the indentation of turnover lines. 
When the reference list is completed the command \verb"\end{thereferences}" 
or \verb"\end{thebibliography}" should be 
used to end the reference environment as appropriate.

The References should be formatted according to a consistent 
style and that outlined here is highly recommended; 
it will save a considerable amount of time in copy editing and correction
if the references are prepared in an acceptable style. 
If authors wish to use a different
style for references they should discuss the proposed format with the 
commissioning staff at the earliest possibility.
Brief descriptions of the use of the two 
referencing systems are given below.

\section{Harvard system}
In the Harvard alphabetical 
system\index{references, Harvard alphabetical system} 
the name of the author appears in the text together 
with the year of publication. 
As appropriate, either the date or the name 
and date are included within parentheses. Where there are only two authors 
both names should be given in the text; if there are more than two 
authors only the first name should appear followed by `{\it et al}' 
(which is obtained by 
typing \verb"\etal"). When two or 
more references to work by one author or group of authors occur for the 
same year they should be identified by including a, b, etc after the date 
(e.g.\ 1986a). If several references to different pages of the same article 
occur the appropriate page number may be given in the text, e.g.\ Kitchen 
(1982, p~39).


The reference list consists of an 
alphabetical listing by authors' names and in date order for each 
author or group of identical authors. Each individual reference must 
be 
typed as a separate paragraph.

There will be two basic types of 
entries within the reference list: (i) those to journal articles and 
(ii) those to books, conference proceedings and reports. For both of 
these types of references font changes are required; in the 
subsections below we describe these font changes.

\subsection{References to journal articles}
Normal references to  
journal articles\index{journal articles} contain three changes of 
font:
the authors and date appear in roman type, the journal title in 
italic, the volume number in bold and the page numbers in roman again. 
A typical journal entry would be:
\begin{thereferences}
\item Cisneros A 1971 {\it Astrophys. Space Sci.} {\bf 10} 87
\end{thereferences}
which is obtained by typing
\begin{verbatim}
\item Cisneros A 1971 {\it Astrophys. Space Sci.} {\bf 10} 87
\end{verbatim}

Features to note are the following.
\begin{itemize}
\item The authors should be in the form: surname (with only the first 
letter capitalized) {\bf followed} by the initials with {\bf no} 
periods after the initials. Authors should be separated by a comma 
except for the last two which should be separated by `and' with no 
comma preceding it. Titles of articles may be included in the  
reference list, if desired, in which case 
the title should be in lower case, except for an initial 
capital, and should follow the date.

\item The journal is in italic and is 
abbreviated\index{references, journal abbreviations}. If a journal has 
several parts denoted by different letters the part letter should be 
inserted after the journal in roman type, e.g.\ \PR A. An 
exception to this is \PL where the part letter is 
included in the volume number. Several frequently referenced journals 
have been assigned a control sequence that typesets the journal name, 
in italic and leaves a space after it. 
Thus \verb"\NP" expands to \NP, and \verb"\JMP" 
expands to \JMP. An italic correction is included 
where necessary. Tables of the available journal abbreviations are 
given in appendix~\ref{jlabs}.

\item The volume number is bold; the page number is roman. Both the 
initial and final page numbers should be given where possible. The 
final page number should be in the shortest possible form and 
separated from the initial page number by an en rule (\verb"--"), e.g.\ 
1203--14.

\item Where there are two or more references with identical authors, 
the authors' names should not be repeated but should be replaced by 
\verb"\dash" on the second and following occasions. Thus
\begin{verbatim}
\item Harrison P G and Willett M J 1988 {\it Nature\/} 
  {\bf 332} 337--9
\item\dash 1989 {\it J. Chem. Soc. Faraday Trans.} I {\bf 85} 
  1907--20
\end{verbatim}
\end{itemize}


\subsection{References to books, conference proceedings and reports}
References to books\index{references, books},
proceedings\index{references, proceedings}, reports\index{references,
reports} and preprints\index{references, preprints} are similar, but have
only two changes of font. The authors and date of publication are in
roman, the title of the book is in italic, and the editors, publisher,
town of publication and page number are in roman. A typical reference to a
conference and a book are
\begin{thereferences}
\item Caplar R 1973 {\it Proc. Int. Conf. on Nuclear Physics, 
 Munich} vol~1 (Amsterdam: North-Holland) p~517
\item Dorman L I 1975 {\it Variations of Galactic Cosmic Rays} 
(Moscow: Moscow State University Press) p~103
\end{thereferences}     
which would be obtained by typing
\begin{verbatim}
\item Caplar R 1973 {\it Proc. Int. Conf. on Nuclear Physics, 
 Munich} vol~1 (Amsterdam: North-Holland) p~517
\item Dorman L I 1975 {\it Variations of Galactic Cosmic Rays} 
(Moscow: Moscow State University Press) p~103
\end{verbatim}


Features to note are the following:
\begin{itemize}
\item Book titles are in italic and should be spelt out in full with 
initial capital letters for all except minor words. For conference
proceedings words such as 
Proceedings, Symposium, International, Conference, Second, etc should 
be abbreviated to Proc., Symp., Int., Conf., 2nd, 
respectively, but the rest of the title should be given in full, 
followed by the town or city where the conference was held. For 
Laboratory Reports the Laboratory should be spelt out wherever 
possible, e.g.\ {\it Argonne National Laboratory Report}.

\item The volume number as, for example, vol~2, should be followed by 
the editors, if any, in a form such as ed~A~J~Smith and P~R~Jones. Use 
\etal if there are more than two editors. Next comes the town of 
publication and publisher, within parentheses and separated by a colon, 
and finally the page numbers preceded by `p' if only one number is given 
or `pp' if both the initial and final numbers are given. The page 
numbers should be tied (with a \verb"~") to the p or pp.
\end{itemize}

The cross referencing facilities of \LaTeX\ are not really useful if
Harvard style referencing is being used. This is because the reference
list is alphabetical and so additional references can simply be inserted
in the list without any renumbering being required and because 
the form of
citation in the text varies according to context.

\section{Vancouver system}
In the Vancouver system references are numbered sequentially 
throughout the text\index{references, Vancouver numerical system}. 
The numbers occur within square brackets and one 
number can be used to designate several references. The reference list 
is in numerical, not alphabetical, order. The number 
of the reference appears in square brackets, and each 
individual reference should to be typed on a separate line. 
The reference 
list {\bf must} begin with the command \verb"\begin{thebibliography}{#1}", 
where \verb"#1" 
is any number with the same number of digits as the highest reference 
number used. 

As in the Harvard system, there will be two basic types of entries 
within the reference list: (i) those to journal articles and (ii) 
those to books, conference proceedings and reports. For both of these 
types of references font changes are required; the method is exactly 
as was described in the previous subsection. In fact, references to 
journals and books are very similar to those in the Harvard system. 
There are two major differences. Firstly, the reference number
must appear at the start of a reference by using \verb"\bibitem"
instead of \verb"\item" at the start of a new reference. Secondly, 
when two or more separate references with identical authors occur, the 
author names are spelt 
out in full, i.e.\ they are {\bf not} replaced with \verb"\dash". A typical 
numerical reference list might begin
\begin{thebibliography}{9}
\bibitem{dorm} Dorman L I 1975 {\it Variations of Galactic Cosmic Rays} 
(Moscow: Moscow State 
University Press) p~103
\bibitem{casp} Caplar R and Kulisic P 1973 {\it Proc. Int. Conf. on Nuclear 
Physics, Munich} vol~1 (Amsterdam: North-Holland) p~517
\bibitem{cis} Cisneros A 1971 {\it Astrophys. Space Sci.} {\bf 10} 87
\end{thebibliography}
which would be obtained by typing
\begin{verbatim}
\bibitem{dorm} Dorman L I 1975 {\it Variations of Galactic Cosmic Rays} 
(Moscow: Moscow State University Press) p~103
\bibitem{casp} Caplar R and Kulisic P 1973 {\it Proc. Int. Conf. on 
Nuclear Physics, Munich} vol~1 (Amsterdam: North-Holland) p~517
\bibitem{cis} Cisneros A 1971 {\it Astrophys. Space Sci.} {\bf 10} 87
\end{verbatim}

Blank lines between each reference are not necessary as a new paragraph
is started each time a \verb"\bibitem{}" command is used.
If one number covers more than one reference, and the authors are the 
same, then separate the references by a semicolon and type the second 
reference after the first without a paragraph break. For example:
\begin{verbatim}
\bibitem{} Bertini R 1984 \PL {\bf 136B} 29; 1985 \PL {\bf 158B} 19
\end{verbatim}
If the year and journal are also the same these need not be 
repeated.


If one number covers more than one reference, and the 
authors are different, then the second reference should appear
on a separate line but without a number, this is achieved by replacing the 
\verb"\bibitem{}" command with \verb"\item[]" or \verb"\nonum". 
For example:
\begin{verbatim}
\bibitem{fesh} Feshbach H 1986 \PL {\bf 168B} 318
\nonum Fano U 1961 \PR {\bf 124} 1866
\bibitem{bert} Bertini R 1984 \PL {\bf 136B} 29
\end{verbatim}
which gives
\begin{thebibliography}{9}
\bibitem{fesh} Feshbach H 1986 \PL {\bf 168B} 318
\item[] Fano U 1961 \PR {\bf 124} 1866
\bibitem{bert} Bertini R 1984 \PL {\bf 136B} 29
\end{thebibliography}

If \LaTeX\ cross referencing\index{cross referencing, citations}
is being used then the label for the 
reference is included within the brace following \verb"\bibitem"
and \verb"\cite{label}" is used to refer to the reference in the 
text, e.g.\ \cite{fesh}. For more details on cross referencing see Lamport 
(1986).



\section{Reference lists}
A complete reference list should provide the reader with enough information 
to locate the article concerned and should consist of: name(s) and 
initials, date published, title of journal or book, volume number, 
editors, if any, and, for books, town of publication and publisher in 
parentheses, and finally the page numbers. Titles of journal articles may 
also be included. Up to ten authors may be given in the reference 
list; where there are more than ten only the first should be given 
followed by `{\it et al}'. The terms {\it op.\ cit.}, {\it loc.\ cit.}\ 
and {\it ibid.}\ should not be used. Unpublished conferences and 
reports should generally not be included in the reference list and 
articles in the course of publication should be entered only if the 
form of publication is known. References to preprints should give 
the title of the preprint and/or preprint number (if relevant). A 
thesis submitted for a higher degree may be included in the reference 
list if it has not been superseded by a published paper and is 
available through a library; sufficient information should be given 
for it to be traced readily.  





\chapter{Indexing}
\section{Generating an index file}
An index can be compiled manually, by highlighting words 
throughout the text and then inserting page numbers when the final 
version of the text is ready, or semi-automatically, using the \LaTeX\
indexing facilities. A good index is important; so much so that the author 
of \TeX\ himself has stated (Knuth 1986) that he does not 
believe in completely 
automating the preparation of an index: he puts the final touches to 
his indexes by hand. The \LaTeX\ indexing facilities, 
however, should prove to be  
helpful in preparing a good index. If an index file is to be written 
the command \verb"\makeindex" should be added to the preamble.
This causes a \verb".idx" file  to be written containing information
written by \verb"\index" commands embedded in the text.
The command \verb"\index{physics}" appearing on page 27 causes an 
entry on the index file of 
\verb"\indexentry{physics}{27}". \verb"\index" produces no text and if
there has been no \verb"\makeindex" command it does nothing. For more
details of index generation see Lamport (1986).
The program \verb"makeindex" (\verb"makeindx" on PCs) processes the 
.idx file produced and sorts entries and puts the extra space between 
entries starting with different letters, it writes the output onto a file
with the .ind extension.
Some extra work such as removing repeated main entries and 
subentries, removing inconsistencies and adjusting capitalisation
is required to transform the index file into 
a really useful index. This should normally be left until the book is
complete and final proof corrections done and for books for which the
authoer is not producing camera-ready copy will be done at Institute of
Physics Publishing.

The .ind file generated from makeindex 
includes the commands for index environment\index{index} which
generates the index as a chapter in
double-column 
format. The final index is produced by including 
\verb"\input file.ind" in the master file.
\verb"file.ind" is the edited version of the output file from makeindex.
The index file should consist of \verb"\item"s, \verb"\subitem"s and
\verb"\subsubitem"s, with \verb"\indexspace" inserted each time the 
initial letter
changes. A portion of a typical index file is
\begin{verbatim}
\begin{index}
\item Fluctuation measurements, 249
\item Flux
\subitem function, 145
\subitem tube, 29--31
\indexspace
\item H mode, 186
\item HBTX, 21
\end{index}
\end{verbatim}




\chapter{Procedures}
Authors commissioned to write a book in \LaTeX\ should submit a draft
chapter (both in hard copy and electronic form) 
to the Commissioning Editor as soon as possible after starting writing. 
A Desk Editor will then
evaluate the material sent to check adherence to house style
and compatibility of the author's files with our production system.
If problems can be eliminated at this stage it will ensure that the final
manuscript can be processed without problems or delays.
It is also possible for authors to meet an editor if 
there are any queries which could usefully be addressed by such a meeting. 
Staff at IOP are always willing to advise on production of 
manuscripts, to maintain the greatest understanding possible between author and 
editor.

Authors should note that
it is their responsibility to obtain permission\index{copyright
permissions} to reproduce any
previously published material from the author and publisher of the
material.
Once the book is complete and has been reviewed and finalized, it will be 
thoroughly copy-edited\index{production procedures}.
Figure sizes will be checked and 
lettering can be added to figures if necessary. 
Any corrections that are required can either be made 
by an editor here, or the book can be returned to the author for correction. 
After corrections have been made and the revised 
version checked, the figures are inserted in their correct positions and the 
whole book sent to the author for final approval. When it has been 
approved the book will be printed.



\appendix

\chapter{Macros for formatting text and mathematics}

\Tref{txtmacs} lists the basic macros for formatting the text and
\tref{mathmacs} those for simplifying the coding of mathematics.

\index{macros, for text}\begin{table}[b]
\caption{Macros available in IOP book style files to format text. 
Parameters within square 
brackets are optional arguments as are asterisks.}
\label{txtmacs}
\begin{center}
\begin{tabular}{@{}ll@{}}
\br
Macro name&Purpose\\
\mr
\verb"\title"&Title of book\\
\verb"\author"&Author(s)\\
\verb"\maketitle"&Generate title page\\
\verb"\tableofcontents"&Generate contents list\\
\verb"\makeindex"&Make index file\\
\verb"\part*{#1}"&Part heading without author's name\\
\verb"\part[#1]{#2}{#3}"&Part heading without address\\
\verb"\Part[#1]{#2}{#3}{#4}"&Part heading with address\\
\verb"\chapter[#1]{#2}"&Chapter or appendix heading\\
\verb"\chapter*[#1]{#2}"&Unnumbered chapter heading\\
\verb"\Chapter[#1]{#2}{#3}"&Chapter or appendix with author\\
\verb"\Chapter*{#1}{#2}"&Unnumbered chapter with author\\
\verb"\section[#1]{#2}"&Section heading\\
\verb"\subsection{#1}"&Subsection heading\\
\verb"\subsubsection{#1}"&Subsubsection heading\\
\verb"\appendix"&Start of appendices\\
\verb"\References"&References heading (new chapter)\\
\verb"\Bibliography"&Bibliography heading (new chapter)\\
\br
\end{tabular}
\end{center}
\end{table}

\newpage

\index{macros, for text}\begin{table}[t]
\addtocounter{table}{-1}
\caption{(Continued.)}
\begin{center}
\begin{tabular*}{27pc}{@{}ll@{}}
\br
Macro name&Purpose\\
\mr
\verb"\begin{thereferences}"&Start Harvard references\\
\verb"\end{thereferences}"&End Harvard references\\
\verb"\dash"&Rule for repeated authors\\
\verb"\begin{thebibliography}{#1}"&Start Vancouver references\\
\verb"\end{thebibliography}"&End Vancouver references\\
\verb"\nonum"&Unnumbered reference\\
\verb"\index"&Entry to be written in index file\\
\verb"\1"&Entry in index with initial capital\\
\verb"\2"&Subentry in index\\
\verb"\begin{theindex}"&Start of index\\
\verb"\end{theindex}"&End of index\\
\verb"\etal"&\etal for text and reference lists\\
\verb"\caption{#1}"&Figure and table caption\\
\verb"\figblank"&Blank page for landscape figure\\
\verb"\tabblank"&Blank page for landscape table\\
\verb"\centre{#1}{#2}"&Heading centred over several columns\\
\verb"\crule{#1}"&Rule centred over several columns\\
\verb"\br, \mr"&Bold and medium rules for tables\\
\verb"\ns"&Small negative space for use in table\\
\verb"\bs, \ms"&Medium and small interline spaces\\
\verb"\tabnote{#1}"&Table note\\
\verb"\0"&Space of a digit (for table alignment)\\
\verb"\m"&In tables, left overhanging minus sign\\
\verb"\phm"&In tables, phantom minus sign\\
\verb"\pt(#1)"&To get letter within (\dots) italic\\
\verb"\cite{#1}"&Cross referenced citation\\
\verb"\ref{#1}"&Cross reference\\
\verb"\cref{#1}, \Cref{#1}"&Chapter cross references\\
\verb"\sref{#1}, \Sref{#1}"&Section cross references\\
\verb"\eref{#1}, \Eref{#1}"&Equation cross references\\
\verb"\fref{#1}, \Fref{#1}"&Figure cross references\\
\verb"\tref{#1}, \Tref{#1}"&Table cross references\\
\br
\end{tabular*}
\end{center}
\end{table}

\newpage



\index{macros, for mathematics}\index{mathematics, macros for}%
\begin{table}[t]
\caption{Macros available using the AMS fonts and iopams style file}
\label{mathmacs}
\begin{center}
\begin{tabular}{@{}ll@{}}
\br
Macro name&Purpose\\
\mr
\verb"\bi{#1}"&Bold italic for vectors\\
\verb"\Bbb{#1}"&Black board bold (only with AMS fonts)\\
\verb"\bGamma...\bOmega"&Bold upright greek capitals\\
\verb"\bitGamma...\bitOmega"&Bold italic greek capitals\\
\verb"\balpha...\bvarphi"&Bold greek lower case\\
\verb"\bfeta"&Bold lower case eta (not $\backslash${\tt beta}!)\\
\verb"\bell"&Bold $\ell$\\
\verb"\bpartial"&Bold partial\\
\verb"\bimath"&Bold dotless i\\
\verb"\bjmath"&Bold dotless j\\
\verb"\bnabla"&Bold nabla\\
\verb"\bdot"&Bold raised dot (for dot products)\\
\br
\end{tabular}
\end{center}
\end{table}



\chapter{Macros for journal abbreviations}\label{jlabs}

\vspace*{-4pc}\index{journals abbreviations}%
\begin{table}[b]
\caption{Abbreviations for the IOP journals.}
\begin{center}
\begin{tabular}{@{}lll@{}}
\br
Macro name&Short form of journal title&Years relevant\\
\mr
\verb"\CQG"&Class. Quantum Grav.\\
\verb"\IP"&Inverse Problems\\
\verb"\JPA"&J. Phys. A: Math. Gen.\\
\verb"\JPB"&J. Phys. B: At. Mol. Phys.&1968--1987\\
\verb"\jpb"&J. Phys. B: At. Mol. Opt. Phys.&1988 and onwards\\
\verb"\JPC"&J. Phys. C: Solid State Phys.&1968--1988\\
\verb"\JPCM"&J. Phys: Condens. Matter&1989 and onwards\\
\verb"\JPD"&J. Phys. D: Appl. Phys.\\
\verb"\JPE"&J. Phys. E: Sci. Instrum.&1968--1989\\
\verb"\JPF"&J. Phys. F: Met. Phys.\\
\verb"\JPG"&J. Phys. G: Nucl. Phys.&1975--1988\\
\verb"\jpg"&J. Phys. G: Nucl. Part. Phys.&1989 and onwards\\
\verb"\MST"&Meas. Sci. Technol.&1990 and onwards\\
\verb"\NET"&Network\\
\verb"\NL"&Nonlinearity\\
\verb"\NT"&Nanotechnology\\
\verb"\PAO"&Pure and Applied Optics\\
\verb"\PMB"&Phys. Med. Biol.\\
\verb"\PSST"&Plasma Sources Sci. Technol.\\
\verb"\QO"&Quantum Opt.\\
\verb"\RPP"&Rep. Prog. Phys.\\
\verb"\SST"&Semicond. Sci. Technol.\\
\verb"\SUST"&Supercond. Sci. Technol.\\
\verb"\WRM"&Waves in Random Media\\
\br
\end{tabular}
\end{center}
\end{table}

\newpage

\begin{table}[t]
\caption{Abbreviations for some more common non-IOP journals.}
\begin{center}
\begin{tabular}{@{}ll@{}}
\br
Macro name&Short form of journal\\
\mr
\verb"\AC"&Acta Crystallogr.\\
\verb"\AM"&Acta Metall.\\
\verb"\AP"&Ann. Phys., Lpz\\
\verb"\APNY"&Ann. Phys., NY\\
\verb"\APP"&Ann. Phys., Paris\\
\verb"\CJP"&Can. J. Phys.\\
\verb"\JAP"&J. Appl. Phys.\\
\verb"\JCP"&J. Chem. Phys.\\
\verb"\JJAP"&Japan. J. Appl. Phys.\\
\verb"\JMMM"&J. Magn. Magn. Mater.\\
\verb"\JMP"&J. Math. Phys.\\
\verb"\JOSA"&J. Opt. Soc. Am.\\
\verb"\JP"&J. Physique\\
\verb"\JPhCh"&J. Phys. Chem.\\
\verb"\JPSJ"&J. Phys. Soc. Japan\\
\verb"\JQSRT"&J. Quant. Spectrosc. Radiat. Transfer\\
\verb"\NC"&Nuovo Cimento\\
\verb"\NIM"&Nucl. Instrum. Methods\\
\verb"\NP"&Nucl. Phys.\\
\verb"\PF"&Phys. Fluids\\
\verb"\PL"&Phys. Lett.\\
\verb"\PR"&Phys. Rev.\\
\verb"\PRL"&Phys. Rev. Lett.\\
\verb"\PRS"&Proc. R. Soc.\\
\verb"\PS"&Phys. Scr.\\
\verb"\PSS"&Phys. Status Solidi\\
\verb"\PTRS"&Phil. Trans. R. Soc.\\
\verb"\RMP"&Rev. Mod. Phys.\\
\verb"\RSI"&Rev. Sci. Instrum.\\
\verb"\SSC"&Solid State Commun.\\
\verb"\SPJ"&Sov. Phys.--JETP\\
\verb"\ZP"&Z. Phys.\\
\br
\end{tabular}
\end{center}
\end{table}



\References

\begin{thereferences}

\item[] American Mathematical Society 1990 {\it AMS-\LaTeX\ Version 1.1
User's Guide} (Providence, RI: American Mathematical Society)

\item[] American Mathematical Society 1991 {\it AMSFONTS Version 2.1
User's Guide} (Providence, RI: American Mathematical Society)


\item[] Botway L and Biemesderfer C 
{\it \LaTeX\ Command Summary} (Providence, RI:
\TeX\ User's Group)                  

\item[] Goossens M, Mittelbach F and Samarin A 1994 {\it The \LaTeX\
Companion} (Reading, Ma: Addison-Wesley)

\item[] Knuth D E 1986 {\it The \TeX book} (Reading, MA: Addison-Wesley)

\item[] Lamport L 1986 {\it \LaTeX: A Document Preparation System} (Reading, MA: Addison-Wesley)

\item[] Urban M 1990 {\it An Introduction to \LaTeX} (Providence, RI:
\TeX\ User's Group)

\end{thereferences}


\input lauinstr.ind
\end{document}

 

