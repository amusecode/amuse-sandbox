
%\begin{problem}{}
\subsection{Wide non-interacting binary}\label{Sect:Ass.GD_SE.Widebinary}

dr. Erik Holmberg was working as a student of Knut Lundmark at Lund
Observatory in Sweden where he had access to a number of photoray
tubes. Using these he performed an experiment adopting the principle
that light falls off in identical fashion as gravity. He decided to use
this quality of light to perform a simulation of two colliding
galaxies.

\begin{figure}[htbp] 
  \begin{center}
%      \epsfig{figure=./MonolythicSolvers/fig/EHolmberg.eps,width=0.2\columnwidth}
      \epsfig{figure=./fig/EHolmberg.eps,width=0.2\columnwidth}
      \caption{Erik Holmberg, 1908-2000\newline 
          He discovered that
          colors of paired galaxies are closely correlated, which is
          now called the "Holmberg effect".  
      \label{Fig:1941ApJ....94..385H_fig3}
      }
  \end{center}
\end{figure}

The initial setup for a single 'galaxy' in Holmberg's experiment is
presented in Fig.\,\ref{Fig:1941ApJ....94..385H_fig3}.  This
experiment made him one of the founders of a large body of
astronomical research, much of which is reviewed in
\cite{1987gady.book.....B}. According to some astronomers
\cite{1987gady.book.....B} is the most important textbook written
about galactic dynamics to date.

\begin{figure}[htbp] 
  \begin{center}
%      \epsfig{figure=./MonolythicSolvers/fig/1941ApJ....94..385H_fig3.eps,width=0.5\columnwidth}
      \epsfig{figure=./fig/1941ApJ....94..385H_fig3.eps,width=0.5\columnwidth}
      \caption{{\small Copy of the original fig.3 from
          \cite{1941ApJ....94..385H} presenting the initial conditions
          for a single galaxy in Holberg's experiment of two colliding
          galaxies. }
      \label{Fig:1941ApJ....94..385H_fig3}
      }
  \end{center}
\end{figure}

The progress of his experiment can be followed in the two pictures
presented in Fig.\,\ref{Fig:1941ApJ....94..385H_fig4} The arrows here
indicate the directions in which both 'galaxies' move with respect to
each other.

\begin{figure}[htbp] 
  \begin{center}
%      \epsfig{figure=./MonolythicSolvers/fig/1941ApJ....94..385H_fig4.eps,width=0.5\columnwidth}
      \epsfig{figure=./fig/1941ApJ....94..385H_fig4.eps,width=0.5\columnwidth}
      \caption{{\small Copy of the original fig.4 from
          \cite{1941ApJ....94..385H} presenting one snapshot of the
          results from his calculations. }
      \label{Fig:1941ApJ....94..385H_fig4}
      }
  \end{center}
\end{figure}

\subsubsection{Assignment}
Generate an AMUSE script to reconstruct the initial conditions from
fig\,\ref{Fig:1941ApJ....94..385H_fig3} from
\cite{1941ApJ....94..385H}, in which the light bulbs are point masses.
Run the single 'galaxy' model and study when the pattern starts to
change substantially. Run the simulation in which two 'galaxies'
encounter each other to reproduce
Fig.\,\ref{Fig:1941ApJ....94..385H_fig4} from
\cite{1941ApJ....94..385H}.

Address the following questions in your report:
\begin{itemize}
\item[$\bullet$] what is the best integrator to use for running Holmberg's model?
\item[$\bullet$] For what time does the isolated 'galaxy' remain unaffected by its own self-gravity?
\item[$\bullet$] What instability appears in the single 'galaxy' model?
\item[$\bullet$] What happens when the two galaxies interact?
\item[$\bullet$] What initial conditions or conclusions must clearly have been
  different in the original paper?
\item[$\bullet$] if you run the same initial realization with the worst
  possible N-body integrator do you still obtain the same answer?
\end{itemize}


%\includeonly{SPZ,/home/spz/tex/lib/bib/references}


