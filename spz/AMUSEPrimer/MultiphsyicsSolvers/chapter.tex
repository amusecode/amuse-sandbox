%%%%%%%
% Simon Portegies Zwart
% Version 2.0 February 2012
%
% The Art of Computational Astrophysics
% Graduate School in Computational Astrophysics
%%%%%%%

%-------------------------------------------------------------------------
%-------------------------------------------------------------------------
\Chapter[Multiphysics solvers]
        {Multiphysics solvers}
        {}
%       {S. F. Portegies Zwart\\x
%         Sterrewacht Leiden
%       }

%-------------------------------------------------------------------------
\section{Multiphysics Solvers}\label{Ch:MultiphysicsSolvers:Introduction}

In many real life problems a range of physical phenomena are somehouw
coupled. Nature has a good sense of doing this, but in the computer it
is often less clear how to assure that the solution of a coupled
problem is sufficiently accurate to be able to do a qualitative study.
In this chapter we discuss the coupling between a number of amuse
modules in to what one could (or maybe rather would like to) call
multiphysics solvers. In the previous chapter we almost encountered
such a coubled problem in the radiative transfer, which requires some
sort of gas distribution and an ionizing source. In that particular
example the physics was not coupled. Here in this chapter we will take
the coupling in the physics into account. 

A word of caution is well placed here, in stating that numerically
there is probably not one unique way to couple different physical
domains to a self consistent solution. In fact, we will not even so
far as claiming that the solution we will acquire is self consistent.
The AMUSE approach lends itself excellently for problems in which the
phsyics is rather weakly coupled, and even in these cases validation
is a dark art, rather than a well documented science.

\section{Gravity with stellar evolution}

Combining gravity with stellar evolution has been done for more than a
quarter century \cite{MNRAS.224..193T.193T}.  As a
consequence we sort of understand the numerical subtleties in coupling
those modules.  The most common way of solving the combination of
stellar evolution and stellar dynamics is by simply stepping from one
code to the other in regular time intervals. Generally the time
interval of choise is set by the dynamics as a fraction of the
crossing time of the star cluster. 

There are two sort of events that make stellar evolution coupled with
gravitational dynamics tricky; one of them is the possibility of
sudden events in the stellar evolution, and the other is the
possibility that two stars approach eachother withing a distance so
close that one could consider them to coalesce. Although we do not
treat any of these here, because they are considered in the later
chapters where we discuss more elaborate code couplings.

Another aspect of stellar evolution and gravitational dynamics which
we will ignore here is the possibility of stable binaries in which one
(or both of the) stars start to affed the evolution of the other. 

What we will discuss here is the simplest possible coupling between
stellar evolution and gravitational dynamics, much in the spirit of
\cite{MNRAS.224..193T.193T}.


After having initialize all variable and having them supplied with
units the initial conditions may be generated.  this has to be done is
a rather speficif order.  In our example snippet we first generate the
initial mass function, then we construct the converter to be able to
convert between si and n-body units.  Only then we construct the
initial density profile, in this case a King model.  As a last step we
assign the masses from the initial mass function to the King-model
distributed bodies. 
\begin{verbatimtab}[8]
    masses = new_salpeter_mass_distribution(N, Mmin, Mmax)
    converter=nbody_system.nbody_to_si(masses.sum(),Rvir)
    bodies = new_king_model(N, W0,convert_nbody=converter)
    bodies.mass = masses
\end{verbatimtab}[8]

It would be nice to be able to first generate the density profile
completely in N-body units and then assign the masses from some mass
function, as is done in {\tt starlab}, but due to the automated unit
conversion it is more practical to have this rather arcane ordering in
the calling sequence for constructing a star cluster. 

As an alternative one could read the initial conditions from a file, like when the file has a starlab format:
\begin{verbatimtab}[8]
    bodies = io.read_set_from_file(filename, 'starlab') 
    converter=nbody_system.nbody_to_si(bodies.particles.masses.sum(),
                                      bodies.virial_radius())
\end{verbatimtab}[8]

Having the inital conditions set up, we can initiate the two codes to
use, one stellar evolution code and a gravitational solver.  The order
in which you declare the various solvers is not important.

In order to be able to communicate information between the stellar
evolution, gravity modules and the framework, we declare three
channels:
\begin{verbatimtab}[8]
    channel_from_gd_to_framework = gravity.particles.new_channel_to(bodies)
    channel_from_se_to_framework = stellar.particles.new_channel_to(bodies)
    channel_from_framework_to_gd = bodies.new_channel_to(gravity.particles)
\end{verbatimtab}[8]
The first and second channel are used to get data from the specific
module to the framework, whereas the third channel allows us to write
to the gravitational dnamics module.

The frist channel allows us to copy data from the gravitational dynamics module:
\begin{verbatimtab}[8]
        channel_from_gd_to_framework.copy()
\end{verbatimtab}[8]
So that we can print the positions of the particle or perform any
postprocessing activities.

We need the masses of the stars from the stellar evolution module in
order to adjust the masses in the gravitational dynamics.  For this we
require two channels, one to copy the mass from stellar evolution to
the framework, and one to copy the mass to the gravity module.

\begin{verbatimtab}[8]
        channel_from_gd_to_framework.copy()
        channel_from_se_to_framework.copy_attributes(["mass"])
        channel_from_framework_to_gd.copy_attributes(["mass"])
\end{verbatimtab}[8]
We do not need a channel from the framework to the stellar evolution
because in our simple script the stellar evolution operates
independently of the gravitational dynamics.

In principle we could have constructed a single channel from the
stellar module directly to the gravity module, but in parctice the
data stream would still go through the framework. This is neede in
order to warrant the proper unit conversion from the stellar masses is
si and the gravitational masses in n-body units.  From a point of view
contrcuting a channel between stellar and gravity directly would
reduce the communication overhead with a factor of 2.

\subsection{Improving the combined solver} 

\subsection{Experiment with the combined solver}

As an experiment we will run an extremely wide binary.



\section{Bridging gravity with gravity}

An often heared complaint about gravitational N-body solvers is their
unfavorite scaling of the compute time as a function of N. this is one
of the resasons for the popularity of gravitational tree codes, like
the Barnes-Hut algorithm \cite{1986Natur.324..446B}.  With direct
summation one could, on a modern workstation perform simulations of a
few to several thousand particles, with a treecode one could go to
several hundred thousand particles.

Direct summation in gravitational solvers is required for many type of
simulations, like planetary system integration and star cluster
studies. Once, these objects are considered in a larger environment,
like a star cluster in a galaxy, one should probably not be
particularly interested in the very high accuracy of the direct
summation method for rsolving the tidal field of the cluster in the
Galaxy. In particular when one is not interested in the response of
the galaxy on the internal dynamics and evolution of the cluster. Very
often one is only interested in the effect of the galaxy on the
cluster. 

When the time and spatial scales are reasonably separated one can
adopt a hamiltonian splitting of the equations of motion. In these
cases a kick-drift-kick mechanism, as in the good old Verlet-Leapfrog
integrator can be adopted to integrate the direct N-body solver as an
embedded object in the less accurate treecode.

\subsection{Bridge}

In the classical bridge scheme, \cite{2007PASJ...59.1095F} consider a
star cluster orbiting a parent galaxy.  The cluster is integrated
using accurate direct summation of the gravitational forces among all
stars.  Interactions among the stars in the galaxy, and between
galactic and cluster stars, are computed using a hierarchical tree
force evaluation method \cite{1986Natur.324..446B}.

In bridge, the Hamiltonian of the entire system is divided into two
parts:
\begin{equation}
        H  =  H_A + H_B,
\end{equation}
where $H_A$ is the potential energy of the gravitational interactions
among galaxy particles ($W_g$) and between galaxy particles
and the star cluster ($W_{g-c}$):
\begin{equation}
        H_A  =  W_g + W_{g-c},
\end{equation}
and $H_B$ is the sum of the total kinetic energy of all particles
($K_g + K_c$) and the potential energy of the star cluster particles
($W_c$):
\begin{equation}
        H_B  =  K_g + K_c + W_c.
\end{equation}
The time evolution of any quantity $f$ under this Hamiltonian can then
be written approximately, because we have truncated the formal
solution to just the second-order Leapfrog terms, as:
\begin{equation}
        f^\prime (t+\Delta t) = e^{\half \Delta t A} e^{\Delta t B}
						  e^{\half \Delta t A} f(t),
\end{equation}
where the operators $A$ and $B$ are defined by $Af = \{f,H_A\}$ and
$Bf = \{f,H_B\}$, and $\{.,.\}$ is a Poisson bracket.  This is the
familiar second-order Leapfrog algorithm.  The evolution can be
impemented as a kick-drift-kick scheme, as illustrated in
Fig.\,\ref{Fig:bridge}.

\begin{figure}[h] 
%\psfig{figure=../fig/bridge.eps,width=\columnwidth}
\caption[]{Schematic kick-drift-kick procedure for the generalized 
           mixed-variable symplectic method \cite{1991AJ....102.1528W}.
           \label{Fig:bridge} }
\end{figure}

\steve{A few comments here.  First, this discussion is pure classical
  mechanics, and I think it will be incomprehensible to most readers.
  I have reworded a bit and added a definition of the Poisson bracket
  operators, which were undefined and unexplained, but I really think
  this goes too far.  The figure tells almost the whole story.  Also,
  we should make clear that the example is for a particular coupled
  dynamical system, but the bridge approach is used in AMUSE as a
  general means of coupling modules.}

\subsection{Implementation of the bridge}

\begin{verbatimtab}[8]
    def evolve_model(self, tend,timestep=None):
        while self.time < tend:    
            dt=min(timestep,tend-self.time)
            self.kick_systems(dt/2)   
            self.drift_systems(self.time+dt)
            self.time+=dt
            self.kick_systems(dt/2)
            self.time=self.time+dt
        return 0    
\end{verbatimtab}[8]

\begin{verbatimtab}[8]
def kick_system(system, get_gravity, dt):
    parts=system.particles.copy()
    ax,ay,az=get_gravity(parts.radius,parts.x,parts.y,parts.z)
    parts.vx=parts.vx+dt*ax
    parts.vy=parts.vy+dt*ay
    parts.vz=parts.vz+dt*az
    channel=parts.new_channel_to(system.particles)
    channel.copy_attributes(["vx","vy","vz"])   
\end{verbatimtab}[8]


\begin{verbatimtab}[8]
    def drift_systems(self,tend):
    self.evolve_model(tend)
\end{verbatimtab}[8]

\subsection{A star cluster in galactic potential}

We can now write the script to simulate a star cluster in a static
background potential of the galaxy.  We first construct a gravity
solver that can provide the potential at any point in the Galaxy. We
require this function to be able to bridge with the N-body code.

We then generate the initial conditions for the cluster and initialize
the N-body integrator.

The bridge is started by constructing a new {\tt gravity} solver which
now contains two codes. The syntax for the bridge is somewhat arcane
at first but we add a {\tt cluster} to the gravity (bridge) solver
which will getting its kicks from the {\tt MilkyWay\_Galaxy}.  The
comma after the {\tt MilkyWay\_Galaxy} is to indicate that it is a
list of classes that affect the {\tt cluster}, even though it is only
one external kick-system that affects the cluster in this case.
\begin{verbatimtab}[8]
    gravity = bridge.Bridge(use_threading=False)
    gravity.add_system(cluster_gravity, (MilkyWay_galaxy(),) )
\end{verbatimtab}[8]

In the event loop we now call the {\tt evolve\_model} from the bridged
{\tt gravity}:
\begin{verbatimtab}[8]
        gravity.evolve_model(time)
\end{verbatimtab}[8]
After this we can get some data from the cluster via a channel and
dump those to a file for later analysis.  Much of the details are
hidden in the {\tt Bridge} class.

\subsection{Experiment with bridged N-body}

\subsection{example from the literature: Do not do this at home}

Rieder etal 2012.... bridged Cosmogrid...

\section{Bridging gravity with gravity II}

Instead of bridging a star cluster simulation with a fixed background
potential we can easily replace the galaxy with a life evolving galaxy
by birdging with another gravity solver rather than with a static
background potential.  For the Galaxy one probably do not want to use
a direct N-body solver, but rather a treecode or something else.



\subsection{Experiment with bridged life gravity}

\section{Bridging gravity with hydrodynamics}
\subsection{Experiment with gravity and hydrodynamics}

\section{coupling hydrodynamics with radiative transfer}
\subsection{Experiment with hydrodynamics and radiative transfer}



\subsection{Hierarchical code coupling}


%\input /home/spz/latex/lib/bib/references
